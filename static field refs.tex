\documentclass[aps,pra,groupedaddress,showpacs,preprint,doubled-space]{revtex4}
\usepackage{graphicx,amsmath,subfigure,amssymb}% Include figure files
\usepackage{dcolumn}% Align table columns on decimal point
\usepackage{bm}% bold math
\usepackage{latexsym}% Include special symbols\
\begin{document}
\title{{\bf The Spectroscopy of Barium $6p_{1/2}nk$ Autoionizing States in a Weak Electric Field}}
\author{J. Nunkaew and T. F. Gallagher}
\affiliation{Department of Physics, University of
Virginia,Charlottesville, VA 22904}
\date{\today}
\begin{abstract}
\label{abst}
We have studied the Ba $6p_{1/2}nk$, $n=17$,
$5\leq k\leq n-1$ Stark autoionizing states by isolated
core excitation (ICE) the Ba $6snk$ Stark bound states in very weak electric fields, an unexplored regime. In zero field the broad $\ell=5$ state overlaps all the higher $\ell$ states, and in fields less than xx V/cm the Stark states composed of the $\ell>5$ states are within the zero field width of the $nh$ state. In the ICE spectra the strongest transition from the $6snk$ Stark stae is the the $6p_{1/2}nk$ Stark state, and shakeup satellite transitions are observed to higher lying Stark states but not to lower lying Stark states, a phenomenon which we attribute to the fact that the two Stark manifolds are both incomplete and different. The observed spectra can be described by treating the problem as discrete states, the Stark sattes composed of the $\ell>5$ states, embedded in one continuum state, the $6p_{1/2}nh$ state.
\end{abstract}
\pacs{}
\maketitle


\begin{thebibliography}{99}
\bibitem{Johnsson} P. Johnsson, J. Mauritsson, T. Remetter, A. L'Huillier, and K. J. Shafer, Phys. Rev. Lett. {\bf 99}, 233001 (2007). https://journals.aps.org/prl/abstract/10.1103/PhysRevLett.99.233001

\bibitem{Tong} X. M. Tong, P. Ranitovic, C. L. Cocke, and N. Toshima, Phys. Rev. A. {\bf 81}, 021404(R)(2010). http://iopscience.iop.org/article/10.1088/1367-2630/12/1/013008/meta

\bibitem{Krausz} F. Krausz, and M. Ivanov, Rev. Mod. Phys. {\bf 81}, 163 (2009). https://journals.aps.org/rmp/abstract/10.1103/RevModPhys.81.163

\bibitem{Mauritsson} J. Mauritsson, P. Johnsson, E. Gustafson, A. L'Huilier, K. J. Shafer, and M. B. Gaarde, Phys. Rev. Lett. {\bf 97}, 178 (1978). https://journals.aps.org/prl/abstract/10.1103/PhysRevLett.97.013001


\bibitem{Singh} K. P. Singh, F. He, P. Ranitovic, W. Cao, S. De, D. Ray, S. Chen, U. Thumm, A. Becker, M. M. Murnane, H. C. Kapteyn, I. V. Litvinyuk, and C. L. Cocke, Phys. Rev. Lett. {\bf 104}, 023001 (2010). https://journals.aps.org/prl/abstract/10.1103/PhysRevLett.104.023001

\bibitem{Carrat} V. Carrat, E. Magnuson, and T. F. Gallagher, Phys. Rev. A {\bf 92), 063414 (2015). https://journals.aps.org/pra/abstract/10.1103/PhysRevA.92.063414

\bibitem{Shuman} E. S. Shuman, R. R. Jones, and T. F. Galalgher, Phys. Rev. Lett. {\bf 101}, 263001 (2008). https://journals.aps.org/prl/abstract/10.1103/PhysRevLett.101.263001

\bibitem{Littman} M. G. Littman and H. J. Metcalf, Appl. Opt. {\bf 17}, 2224 (1978). https://www.osapublishing.org/ao/abstract.cfm?uri=ao-17-14-2224

\bibitem{Hansch} T. W. Hansch, Appl. Opt. {\bf 11}, 895 (1972). https://www.osapublishing.org/ao/abstract.cfm?uri=ao-11-4-895


\end{thebibliography}
\end{document}
