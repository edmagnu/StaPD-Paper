%% ****** Start of file apstemplate.tex ****** %
%%
%%
%%   This file is part of the APS files in the REVTeX 4 distribution.
%%   Version 4.1r of REVTeX, August 2010
%%
%%
%%   Copyright (c) 2001, 2009, 2010 The American Physical Society.
%%
%%   See the REVTeX 4 README file for restrictions and more information.
%%
%
% This is a template for producing manuscripts for use with REVTEX 4.0
% Copy this file to another name and then work on that file.
% That way, you always have this original template file to use.
%
% Group addresses by affiliation; use superscriptaddress for long
% author lists, or if there are many overlapping affiliations.
% For Phys. Rev. appearance, change preprint to twocolumn.
% Choose pra, prb, prc, prd, pre, prl, prstab, prstper, or rmp for journal
%  Add 'draft' option to mark overfull boxes with black boxes
%  Add 'showpacs' option to make PACS codes appear
%  Add 'showkeys' option to make keywords appear
\documentclass[aps,pra,preprint,groupedaddress]{revtex4-1}
%\documentclass[aps,prl,preprint,superscriptaddress]{revtex4-1}
%\documentclass[aps,prl,reprint,groupedaddress]{revtex4-1}

% You should use BibTeX and apsrev.bst for references
% Choosing a journal automatically selects the correct APS
% BibTeX style file (bst file), so only uncomment the line
% below if necessary.
%\bibliographystyle{apsrev4-1}

\begin{document}

% Use the \preprint command to place your local institutional report
% number in the upper righthand corner of the title page in preprint mode.
% Multiple \preprint commands are allowed.
% Use the 'preprintnumbers' class option to override journal defaults
% to display numbers if necessary
%\preprint{}

%Title of paper
\title{Phase Dependent Ionization of Rydberg Atoms in Static Fields}

% repeat the \author .. \affiliation  etc. as needed
% \email, \thanks, \homepage, \altaffiliation all apply to the current
% author. Explanatory text should go in the []'s, actual e-mail
% address or url should go in the {}'s for \email and \homepage.
% Please use the appropriate macro foreach each type of information

% \affiliation command applies to all authors since the last
% \affiliation command. The \affiliation command should follow the
% other information
% \affiliation can be followed by \email, \homepage, \thanks as well.
\author{Eric Magnuson}
\email[]{edm5gb@virginia.edu}
\author{Tom Gallagher}
%\homepage[]{Your web page}
%\thanks{}
%\altaffiliation{}
\affiliation{University of Virginia, Department of Physics}

%Collaboration name if desired (requires use of superscriptaddress
%option in \documentclass). \noaffiliation is required (may also be
%used with the \author command).
%\collaboration can be followed by \email, \homepage, \thanks as well.
%\collaboration{}
%\noaffiliation

\date{\today}

\begin{abstract}
Pump-probe schemes using high frequency pulsed light synchronous to strong field low frequency fields are a prolific tool for probing atomic, molecular and surface electron dynamics. We realize one such system in Rydberg states of Li using an 819-nm excitation laser amplutde modulated synchronously to a 15.9-GHz microwave field. We show that when the modulation is at the same frequency of the microwave field, phase dependent ionization is only observed in the presence of static fields. Our results are well described by a computational model. Analysis of this model shows the importance of multiple classical electron orbits.
\end{abstract}

% insert suggested PACS numbers in braces on next line
\pacs{}
% insert suggested keywords - APS authors don't need to do this
%\keywords{}

%\maketitle must follow title, authors, abstract, \pacs, and \keywords
\maketitle

% body of paper here - Use proper section commands
% References should be done using the \cite, \ref, and \label commands
\section{\label{intro}Introduction}
% Put \label in argument of \section for cross-referencing
%\section{\label{}}
%\subsection{}
%\subsubsection{}

\section{\label{back}Background}

\section{\label{exp}Experimental Methods}

\textbf{\emph{Physical orientation}}
In this experiment, Li atoms in a columated thermal beam are optically excited to high lying Rydberg or continuum states along the path $2s \rightarrow 2p \rightarrow 3d \rightarrow nf, \epsilon f$. These optical beams intersect at a right angle forming a rectangular 1 mm$^3$ excitation region. This region is at an anti-node of a 15.9 GHz Fabry-Perot microwave cavity. The 1-mm$^3$ region is much smaller than the extent of the microwave antinode, allowing us to consider the microwave field constant across the region. The interaction region is enclosed on top, bottom and two sides by aluminum plates. Combined with the two Fabry-Perot mirrors, this forms a 10-cm cubic enclosure. Bias voltages can be applied independently to each plate and mirror to control static fields in the interaction region.

\textbf{\emph{Timing}}
The microwave cavity is allowed to load for 280 ns before the first laser pulse, and the microwave input is shut off 10 ns after the last laser pulse allowing the cavity to empty. Synchronous with the MW power envelope, a static field pulse is applied to the top and bottom aluminum plates to produce a vertical static field in the interaction region. The 610-nm and 671-nm lasers are pulsed synchronously for 20 ns, after which the 819-nm laser is pulsed with a square envelope for 20 ns. One microsecond after the last laser pulse, we field ionize surviving Rydberg states within 100 GHz of the ioniztion limit by applying a negative voltage pulse to an aluminum plate below the interaction region. Ionized electrons are pushed into a microchannel plate (MCP) assembly which produces a voltage pulse proportional to the number of electrons detected. This pulse is integrated through a boxcar integrator and recorded.

\textbf{\emph{Amplutide modulated laser}}
Probing phase dependence in this experiment is achieved by synchronizing the amplitude modulation of the 819-nm laser to a microwave field in the cavity. The laser field can be described by an envelope modulating a fundamental frequency:
\begin{equation}
E_{opt}(t) = E_o \sin{(\omega_o t)} \cos{(\omega(t-t_0))}
\end{equation}
Using an optical delay line, the phase of the modulation envelope of the 819-nm laser can be delayed relative to phase of the microwave field. This modulation is proportional to the excitation-rate to a Rydberg or continuum state, so delaying the modultion envelope is equivalent to changing the phase $\omega t_0$ of the microwave field at which excitation occurs.

\subsection{\label{dye}Dye Lasers}

We use two dye lasers at 670-nm and 610-nm to drive the $2s \rightarrow 2p$ and $2p \rightarrow 3d$ tranitions, respectively. These dye lasers are pumped by a Quantronix Darwin Nd:YLF. The pump laser produces a 30-W, approximately 100-ns FWHM pulses at a 1-kHz repetition rate. Using a pockels cell (PC) and polarizing beam splitter (PBS), the rising edge of the pulse is picked off and split equally between to the 670-nm and 610-nm lasers. A second pockels cell and beamsplitter directs a 20-ns slice from the peak of the pump pulse to a dye amplifier for the 819-nm laser. The long trailing edge of the pump pulse is dumped.

The 670-nm dye laser uses a Littman-style cavity and LDS-698 laser dye dissolved in Ethanol as a lasing medium. The 610-nm uses a H{\"a}nch style cavity with Rhodamine-610 laser dye dissolved in Ethanol. Both lasers have an approximate FWHM of 10-GHz. To minimize unintended ionization from the $3d$ state, both lasers are attenuated to 2 $\mu J$ pulses before being directed to the vacuum chamber.

\subsection{\label{ampmod} Amplitude Modulated 819-nm Laser}

The 819-nm laser is produced rrom two external-cavity diode lasers, a Toptica DL-100 and DL-Pro. They tuned so that they are separated by the microwave frequency, and overlapped on a 50:50 beamsplitter. This overlapping produces a beat-note in the laser intensity at the microwave frequency. One output from the beamsplitter is directed to a high speed photodetector that can detect the beat note and deliver the signal to the phase-locked-loop. The second output from the beamsplitter is directed through an amplification chain and to the interaction region.

The amplitude modulation of the laser is locked at the microwave frequency with a phase-locked-loop (PLL). The locking is driven by a "fast" feedback to the current input of the DL-Pro, and a "slow" feedback to the scan input. To produce an error signal, the amplitude modulation is detected after the 50:50 beamsplitter by a fast photodiode. The DC signal is filtered out, leaving only the AC component. This is amplified and mixed with the microwave source to produce an error signal. The error signal is passed through a variable attenuator and then connected to the Current input of the Toptica DL-100. This achieves a lock between the laser ampltude modulation frequency and the microwave field that lasts for several minutes.

To achieve longer lock times on the order of hours, we use a Toptica PID-110 to drive the scan input of the DL-Pro. The "fast" current lock is primarily lost due to the lasers drifting past the range of the current input to correct. Low-frequency components of the error signal are processed through a PID and fed to the Scan input. This corrects long term drifts in the amplitude modulation frequency, leaving only transient errors for the "fast" loop to correct.

The second output of the 50:50 beamsplitter produces 30 mW of aplitude modulated light. This is passed through a tapered amplifier and a dye amplifier. The Toptica Tapered Amplifier increases the continuous beam power to 800 mW. The dye amplifier is pumped by a 20-ns square pulse picked from the peak of the Nd:YLF laser pulse. We use LDS-819 dissolved in Ethanol as the amplfication medium. This outputs a 20-ns long, 6 $\mu$W pulse of amplitude modulated 819-nm light.

\subsection{\label{cavity} Microwave Apparatus}

A Hittite HMC T-2100 synthesizer tuned to the 15.9-GHz resonance of the microwave cavity is used as our microwave source. The Hittite produces 9 dBm, and a splitter diverts half of the power to the microwave mixer to generate the error signal for the PLL. The other half of the signal is formed into 300-ns pulses by a microwave switch, and then amplified by a Hughes 8020H04F traveling-wave-tube-amplifier (TWTA). Between the TWTA and the cavity, there is a 0 to 50 dBm variable attenuator allowing us to control the intensity of the pulse incident on the microwave cavity.

The microwave cavity is a Fabry-Perot cavity composed to two brass spherical mirrors. These mirrors have a radius of curvature of 10 cm and a 10.2 cm diameter, with a 7.85 cm on axis separation. The 15.9 GHz resonance is the TEM$_{008}$ mode, with a quality of \emph{CAVITY QUALITY}. We are able to determine the field inside the cavity to \emph{15\%}.

\subsection{\label{fields} Static Fields}

This experiment depends on the detection of long lived Rydberg states close to the ionization limit. To prevent these states from ionizing before detection, we must minimize the persistent static field in the interaction region. We accomplish this by surrounding the interaction region on two sides with the brass microwave cavity mirrors, and on the remaining two sides, top, and bottom with polished aluminum plates. A voltage can be applied independently to each plate or mirror, allowing us to compensate persistent static fields in every direction. We measure the depressed ionization limit to minimize stray fields and estimate the residual persistent static field. In this manner, we determine the remaining persistent static field to have a magnitude of 1.5 mV / cm. \emph{ADD REFERENCE}.

To observe phase dependent ionization, we need to apply a pulsed vertical static field to the interaction region during excitation of Rydberg states. This is done using a two-channel arbitrary waveform generator \emph{MODEL} to apply 300-ns square pulses of opposite magnitudes to the top and bottom bias plates. This square pulse is synchronous with the microwave pulse, the leading edge arriving 260-ns before the first laser pulse and the trailing edge arriving 20-ns after the end of the 819-nm laser pulse. Turning off the applied static field minimizes the static field ionization of the high-lying states we wish to observe. 1 $\mu s$ after the final laser pulse, the same AWG applies a voltage to the bottom plate to produce a -0.65 V / cm ionization field, ionizing high-lying states and pushing the electrons toward the MCP stack.

\section{\label{comp}Computational Model}

\section{\label{results}Results}

In \emph{FIGURE} we show that, when the laser intensity is modulated at the same frequency as the microwave field, a phase dependent ionization can only be observed when a static field is present during excitation. This confirms the prediction made by our model, that a static field is needed to lift the vertical symmetry of the system. To produce this result, the central frequency of the excitation laser is tuned to 14-GHz below the depressed ionization limit, in the presence of a 4 V/cm microwave field \emph{CALIBRATION}. The relative phase delay is scanned while measuring the Rydberg signal. Two such scans are completed, one without applying a pulsed static field, one while applying a 14 mV/cm pulsed static field.



\section{\label{disc}Discussion}

\section{\label{conc}Conclusions}

% If in two-column mode, this environment will change to single-column
% format so that long equations can be displayed. Use
% sparingly.
%\begin{widetext}
% put long equation here
%\end{widetext}

% figures should be put into the text as floats.
% Use the graphics or graphicx packages (distributed with LaTeX2e)
% and the \includegraphics macro defined in those packages.
% See the LaTeX Graphics Companion by Michel Goosens, Sebastian Rahtz,
% and Frank Mittelbach for instance.
%
% Here is an example of the general form of a figure:
% Fill in the caption in the braces of the \caption{} command. Put the label
% that you will use with \ref{} command in the braces of the \label{} command.
% Use the figure* environment if the figure should span across the
% entire page. There is no need to do explicit centering.

% \begin{figure}
% \includegraphics{}%
% \caption{\label{}}
% \end{figure}

% Surround figure environment with turnpage environment for landscape
% figure
% \begin{turnpage}
% \begin{figure}
% \includegraphics{}%
% \caption{\label{}}
% \end{figure}
% \end{turnpage}

% tables should appear as floats within the text
%
% Here is an example of the general form of a table:
% Fill in the caption in the braces of the \caption{} command. Put the label
% that you will use with \ref{} command in the braces of the \label{} command.
% Insert the column specifiers (l, r, c, d, etc.) in the empty braces of the
% \begin{tabular}{} command.
% The ruledtabular enviroment adds doubled rules to table and sets a
% reasonable default table settings.
% Use the table* environment to get a full-width table in two-column
% Add \usepackage{longtable} and the longtable (or longtable*}
% environment for nicely formatted long tables. Or use the the [H]
% placement option to break a long table (with less control than 
% in longtable).
% \begin{table}%[H] add [H] placement to break table across pages
% \caption{\label{}}
% \begin{ruledtabular}
% \begin{tabular}{}
% Lines of table here ending with \\
% \end{tabular}
% \end{ruledtabular}
% \end{table}

% Surround table environment with turnpage environment for landscape
% table
% \begin{turnpage}
% \begin{table}
% \caption{\label{}}
% \begin{ruledtabular}
% \begin{tabular}{}
% \end{tabular}
% \end{ruledtabular}
% \end{table}
% \end{turnpage}

% Specify following sections are appendices. Use \appendix* if there
% only one appendix.
%\appendix
%\section{}

% If you have acknowledgments, this puts in the proper section head.
%\begin{acknowledgments}
% put your acknowledgments here.
%\end{acknowledgments}

% Create the reference section using BibTeX:
\bibliography{basename of .bib file}

\end{document}
%
% ****** End of file apstemplate.tex ******

