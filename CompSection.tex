\documentclass[aps,pra,preprint,groupedaddress]{revtex4-1}

\usepackage{graphicx}
\usepackage{amsmath}

\begin{document}

\section{\label{sec:2DMod} Two Dimensional Model}

\subsection{\label{sec:EoM} Equation of Motion}

To analyze these experimental results, we constructed a two-dimensional model of a classical Rydberg electron orbiting in the combined Coulomb, pulsed, and microwave (MW) fields. Expressed in atomic units, the equation of motion is:
\begin{align*}
\ddot{\vec{r}} & = F_{coul}(\vec{r}) - \vec{E}_{P}(t) - \vec{E}_{MW}(t) \\
 & = -\frac{1}{r^2} \cdot \hat{r} - \Phi_P(t) \cdot E_{p} \cdot \hat{z} - \Phi_{MW}(t) \cdot E_{mw} \sin{(\omega t + \phi_0)} \cdot \hat{z} 
\end{align*}
$\Phi_P$ and $\Phi_{MW}$ are envelopes describing the square wave turning off the pulsed field and the exponential ring-down of the MW field:
\begin{align*}
\Phi_P(t \leq t_{off}) & = 1 & \Phi_{MW}(t \leq t_{off}) & = 1 \\
\Phi_P(t > t_{off}) & = 0 & \Phi_{MW}(t > t_{off}) & = e^{-(t-t_{off})/\tau_{MW}}
\end{align*}

% Initial Conditions

The model considers an electron starting from the periapsis of the highly elliptical Rydberg orbit with a set initial energy $W_0$ and initial angular momentum $l_0 = \sqrt{3 \cdot (3+1)}$ corresponding to an $f$ state. These quantities define the initial radius and velocity $r_0, v_0$. For each set of initial conditions, an electron is launched with angular momentum aligned along both $\pm \hat{y}$. The initial orientation of the elliptical orbit is in either in the $\pm \hat{z}$ direction, as defined by the angle of the Laplace-Runge-Lenz vector from $+ \hat{z}$ being $\theta_{LRL} = 0, ~ \pi$. 0 and $\pi$ correspond to electrons launched "uphill" along $\vec{E}_P$, and "downhill" opposite $\vec{E}_P$, respectively

For computation, the vector equations are expressed in terms of Cartesian coordinates \{x,y,z\}. Including the initial conditions, the system of ordinary differential equations to integrate is:
\begin{align*}
x(0) & = \sin{\theta_{LRL}} \cdot r_0 & z(0) & = -\cos{\theta_{LRL}} \cdot r_0 \\
\dot{x}(0) & = \pm \cos{\theta_{LRL}} \cdot v_0 & \dot{z}(0) & = \pm \sin{\theta_{LRL}} \cdot v_0 \quad \text{for} \quad \hat{L} = \pm \hat{y} \\
\ddot{x} & = -\frac{x}{(x^2 + z^2)^{3/2}} & \ddot{z} & = -\frac{z}{(x^2 + z^2)^{3/2}} - \Phi_P(t) \cdot E_P \\
 & & & \quad \quad - \Phi_{MW}(t) \cdot E_{MW} \cdot \sin{(\omega t + \phi_0)}
\end{align*}

\subsection{\label{sec:exe} Execution}

For each initial $e^-$ energy $W_0$ and pulsed field amplitude $E_p$, $e^-$s are launched at 200 initial phases $\phi_0$ between 0 and $2\pi$, with both $\pm \hat{y}$ angular momenta and in the "uphill" and "downhill" directions. The equations of motion are integrated until the MW field has decayed for 5 times the decay constant $t_{off} + 5\tau_{MW}$. After this integration time, the final energy $W_f = 1/2 v_f^2 - 1/r_f$ of the $e^-$ is recorded.

An $e^-$ with a positive final energy is determined to be ionized, while negative final energies are bound and detected. The signal as a function of phase is convolved with the intensity envelope of the AM laser to produce an expected signal. To aid in our analysis, this is done separately for "up" and "downhill" $e^-$, and then the both signals are summed to arrive at the total expected value.

\subsection{\label{sec:res} Results}

Fig.~\ref{fig:2DW0} shows the result for $W_0 = 0$ GHz and $E_P = 0, ~36, ~100$ mV/cm. Experimentally, the peak-to-peak amplitude does not change sign, and we see that in these figures. At small field the "downhill" signal quickly approaches zero due to the depressed ionization limit, and the total signal is dominated by the phase of the "uphill" signal.

\begin{figure}
	\includegraphics{W0_2D}
	\caption{Calculated observed signal from our 2 dimensional model, with initial energy $W_0 = 0$ GHz. The contributions from uphill and downhill electrons are in orange and blue, respectively, with the total expected signal in Green. At $E_{pulse} = 0$ mV/cm, uphill and downhill signals have opposite phase dependence, resulting in a flat total signal. As signal increases, downhill $e^-$ at every launch phase all ionize, and the total signal is dominated by the uphill signal.}
	\label{fig:2DW0}
\end{figure}

Fig.~\ref{fig:2DW20} shows the result for $W_0 = -20$ GHz and $E_P = 0, ~7.2, ~100$ mV/cm. Experimentally, when the AM laser is tuned below the depressed ionization limit the sign of the peak-to-peak amplitude changes sign between small and large $E_P$. This figure accounts for this change. At $E_P = 7.2$ mV/cm, the "downhill" $e^-$ signal has a larger modulation than the "uphill" signal, so the total signal has the "downhill" phase. At $E_P = 100$ mV/cm, downhill $e^-$ are almost completely ionized, and the "uphill" signal dominates the total signal.

\begin{figure}
	\includegraphics{W20_2D}
	\caption{Calculated observed signal from our 2 dimensional model, with initial energy $W_0 = -20 GHz$. Uphill and downhill signals are in blue and orange, respectively, with total signal in green. As in Fig.~\ref{fig:2DW0}, at $E_{pulse} = 0$ mV/cm, the uphill and downhill signals result in a flat total signal. At $E_{pulse} = 7.2$ mV/cm, the downhill signal is diminished, but the total phase dependence increases. The total signal shows a phase dependence shifted by $\pi$ from the small field signal for $W_0 = 0 GHz$ in Fig.~\ref{fig:2DW0}. At $E_{pulse} = 100$ mV/cm, downhill $e^-$ almost all ionize, and the uphill signal dominates the total signal.}
	\label{fig:2DW20}
\end{figure}

\end{document}

